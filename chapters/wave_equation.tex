\chapter{The Wave Equation}

\section{Exercises}

\exercise{1.4}{
	Given positive constants A, a, and a:
	\begin{flalign*}
		\Psi(x,0) =
		\begin{cases}
			A(x/a),       & \text{if}\ 0 \leq x \leq a, \\
			A(b-x)/(b-a), & \text{if } a \leq x \leq b  \\ 0 &
			   \text{otherwise}
		\end{cases}
	\end{flalign*}

	\begin{itemize}
		\item Normalize $\Psi$.
		      \begin{flalign*}
			      \int_{-\infty}^\infty |\Psi(x,t)|^2 ~dx = 1
			      \\
			      A^2\left( \int_0^a \frac{x^2}{a^2} ~dx +
			      \int_a^b \frac{(b-x)^2}{(b-a)^2} ~dx \right) & =
			      1                                                \\
			      A^2\left( \frac{a}{3} + \frac{b-a}{3} \right)
			                                                   & =
			      1 \implies A = \sqrt[2]{\frac{3}{b}}             \\
			      \Psi(x,0) =
			      \begin{cases}
				      \sqrt[2]{\frac{3}{b}}(x/a),       &
				      \text{if}\ 0 \leq x \leq a,         \\
				      \sqrt[2]{\frac{3}{b}}(b-x)/(b-a), &
				      \text{if } a \leq x \leq b          \\ 0 &
				         \text{otherwise}
			      \end{cases}
		      \end{flalign*}
		\item Where is particle most likely to be found at $t = 0$?
		      Based on plots, you will see it is most likely at
		      position a.
		\item Probablility of finding particle to the left of a? Check
		      with b = a and b = 2a.
		      \begin{flalign*}
			      \int_0^a \left|
			      \sqrt[2]{\frac{3}{b}}(x/a)\right|^2 ~dx \\
			      \int_0^a \frac{3}{b}(x^2/a^2) ~dx =
			      \frac{a}{b}
		      \end{flalign*}
		\item What is the first moment (expected value) of x?
		      \begin{flalign*}
			      \langle x \rangle &=\int_0^b x\Psi(x,t) dx = \int_0^a \sqrt[2]{\frac{3}{b}}(x/a) ~dx + \int_a^b  \sqrt[2]{\frac{3}{b}}(b-x)/(b-a) ~dx\\
				  &=\frac{b+2a}{4}
		      \end{flalign*}
	\end{itemize}
}

\exercise{1.5MOD}{
	Given positive, real constants A, $\lambda$, $\omega$:
	\begin{flalign*}
		\Psi(x,t) = Ae^{-\lambda|x|-i\omega t}
	\end{flalign*}
	\begin{itemize}
		\item Normalize $\Psi$.
		\begin{flalign*}
			&\int_{-\infty}^{\infty} |\Psi(x,t)|^2 ~dx =  \int_{-\infty}^{\infty} \Psi^* \Psi = 1\\
			&\int_{-\infty}^{\infty} A^2 e^{-2\lambda |x|} e^{-i\omega t}e^{i\omega t} ~dx =1\\
			A^2 &\int_{-\infty}^{\infty} e^{-2\lambda |x|} ~dx =1\\
			A^2 &\left( \int_{-\infty}^{0} e^{-2\lambda \cdot (-x)}~dx + \int_{0}^{\infty} e^{-2\lambda \cdot (x)} \right) = 1\\
			A^2 &\left( \frac{1}{2 \lambda}e^{2 \lambda x} \right)  \Biggr|_{-\infty}^0 - A^2 \left( \frac{1}{2 \lambda}e^{-2 \lambda x} \right)  \Biggr|_{0}^\infty =1\\
			&\frac{A^2}{\lambda} =1 \implies A = \sqrt[2]{\lambda}\\
			\Psi(x,t) &= \sqrt{\lambda}e^{-\lambda|x|-i\omega t}\\
			|\Psi(x,t)|^2 &= \Psi^* \Psi = \lambda e^{-\lambda|x|-i\omega t} e^{-\lambda|x|+i\omega t} = \lambda e^{-2 \lambda |x|}
		\end{flalign*}
		\item Find the $n^{th}$ moment.
			\begin{flalign*}
				\langle x^n \rangle &= \int_\mathbb{R}  x^n \lambda e^{-2 \lambda |x|} ~dx\\
				&= \lambda \left(  \int_{-\infty}^0 x^n e^{2 \lambda x} + \int_{0}^\infty x^n e^{-2 \lambda x}\right) ~dx
			\end{flalign*}
			Now note the following is smells like the gamma function,
			\begin{flalign*}
				I_{n_1} = \int_{-\infty}^0 x^n e^{2 \lambda x} ~dx = -\int_0^{-\infty} x^n e^{2 \lambda x} ~dx
			\end{flalign*}
			By using substitution of the type $u = -2 \lambda x$ we get,
			\begin{flalign*}
				I_{n_1} = \frac{-1}{2\lambda (-2\lambda)^n} \int_0^\infty e^{-u}u^n du = \frac{(-1)^n}{(2\lambda)^{n+1}} \Gamma(n+1), ~\Re(n) > -1
			\end{flalign*}
			where the last equation can be used to show the required base case of $I_0=\frac{1}{2\lambda}$. A similar analysis for the second integrand gives us the combined relation
			\begin{flalign*}
				\langle x^n \rangle &= \lambda I_n = \lambda(I_{n_1}+I_{n_2}) \\
				& = \lambda \Biggl( \frac{(-1)^n + 1}{(2\lambda)^{n+1}}\Biggr)\Gamma(n+1)
			\end{flalign*}
			For practical purposes, we see that the first few moments give
			\begin{flalign*}
				\langle x \rangle &= 0\\
				\langle x^2 \rangle &= \frac{2 \lambda}{8 \lambda^3}\Gamma(3) = \frac{1}{2 \lambda^2}\\
				\langle x^3 \rangle &= 0 \\
				\langle x^4 \rangle &= \frac{2 \lambda}{32 \lambda^5}\Gamma(4) = \frac{3}{8 \lambda^4}
			\end{flalign*}
			\item Find standard deviation. Compute probability particle is outside one standard deviation from the mean.
			\begin{flalign*}
				\sigma^2 &= \langle x^2 \rangle - \langle x \rangle^2 \implies \sigma = \frac{1}{\lambda \sqrt{2}}\\
				|\Psi(0\pm \sigma, t)|^2 &= |A|^2e^{-2\lambda \sigma}=\lambda e^{-\sqrt{2}}\\
				P_{outside} &= 1-P_{inside} = 1 - \int_{-\sigma}^{\sigma} |\Psi|^2 ~dx =1 - |A|^2 \int_{-\sigma}^{\sigma} e^{-2 \lambda |x|} ~dx = 2 \lambda \int_{\sigma}^{\infty} e^{-2 \lambda x} ~dx = e^{-\sqrt{2}}
			\end{flalign*}
	\end{itemize}
}

\exercise{1.6}{
	Why can't you do integration-by-parts (IBP) directly in the middle expression of Equation 1.29 -- pull the time derivative over into x, note that $\frac{\partial x}{\partial t}= 0$, and conclude that $\frac{ \langle x \rangle }{dx} = 0$?
	 \\ \\Well, you could but this would not allow us to do IBP over some domain D:
	\begin{flalign*}
		\frac{\partial x|\Psi|^2}{\partial t}=\frac{\partial x}{\partial t}|\Psi|^2+x\frac{\partial |\Psi|^2}{\partial t}=x \frac{\partial |\Psi|^2}{\partial t}\\
		\int_{\partial D} x\frac{\partial |\Psi|^2}{\partial t} dx = \int_{\partial D} \frac{\partial (x|\Psi|^2)}{\partial t} dx \neq (x|\Psi|^2) |_{\partial D}
	\end{flalign*}
}

\exercise{1.7}{
	Calculate $\frac{d\langle p \rangle}{dt}$.\\ \\ By Ehrenfest's theorem, expectation values are goverened by classical laws: $\langle p \rangle = m \langle v \rangle = m\frac{d \langle x \rangle}{dt}$.
	Recall the time derivatives for the conjugate pairs or derive it
	yourself. Also note that interchange of differentiaiton to integration
	(Leibnitz integral rule) implicitly assumes the (wave) function and its first
	partial derivative are continuous in time and space (both) in the open
	neighborhood of $\{x\} \times [a,b]$ for any continuous and differentiable
	functions a, b. Text assumes all partials continuous, and by extent
	differentiable (converse not necessarily true). Second order partials assumed continuous for Clairaut's Theorem, $C^2$, throughout text.
	\begin{flalign*}
		\frac{\partial \Psi^* \frac{\partial \Psi}{\partial x}}{\partial t}\\
		&=\pdv{\Psi^*}{t}\pdv{\Psi}{x}+\Psi^* \pdv{\Psi}{x,t}\\
		&=\Bigl(\frac{-i\hbar}{2m}\pdv[2]{\Psi^*}{x}+\frac{iV(x,t)\Psi^*}{\hbar}\Bigr)\pdv{\Psi}{x}+\Psi^*\pdv*{\frac{i \hbar}{2m}\pdv[2]{\Psi}{x}-\frac{iV(x,t)\Psi}{\hbar}}{x}\\
		&= \frac{i \hbar}{2m}\Bigl(\pdv[3]{\Psi}{x}\Psi^*-\pdv[2]{\Psi^*}{x}\pdv{\Psi}{x} \Bigr) + \frac{i}{\hbar} \Bigl(V(x,t)\pdv{\Psi}{x}-\Psi^*\pdv{V(x,t)\Psi}{x}\Bigr)\\
		&= \frac{i \hbar}{2m}\Bigl(\pdv[3]{\Psi}{x}\Psi^*-\pdv[2]{\Psi^*}{x}\pdv{\Psi}{x} \Bigr) + \frac{i}{\hbar} \Bigl(V(x,t)\pdv{\Psi}{x}-\Psi^* V(x,t)\pdv{\Psi}{x}- \Psi^* \pdv{V(x,t)}{x}\Psi\Bigr)\\
		&= \frac{i}{\hbar} \Bigl(|\Psi|^2 \pdv{V(x,t)}{x}\Bigr)\\
	\end{flalign*}
	Whereby we used IBP twice to drop the first term. Accordingly,
	\begin{flalign*}
		\pdv{\langle p \rangle}{t} = -i\hbar \frac{i}{\hbar} \int_{\mathbb{R}} -|\Psi|^2\pdv{V}{x} ~dx = \langle -\pdv{V}{x} \rangle
	\end{flalign*}
	Thus, the time derivative of the expectation value of velocity by mass is equal to the position derivative of the expectation value of potential V.
}\\

\exercise{1.8MOD}{
	Suppose we add a constant $V_0$ to the potential energy. In classical mechanics, this won't change a thing, but what about in quantum mechanics? 
	Show that the function picks up a time-dependent phase factor. What effect does this have on the expectation value of a dynamic variable?
	\\ \\
	\begin{flalign*}
		\pdv{\Psi}{t} = \frac{i\hbar}{2m}\pdv[2]{\Psi}{x}-\frac{i}{\hbar}V(x,t)\Psi(x,t)
	\end{flalign*}
	Set $\zeta(x,t)$ to the wave function holding potential energy $V(x,t)+V_0$ and rewrite to notice a familiar separable PDE,
	\begin{flalign*}
		\pdv{\zeta}{t} &= \frac{i \hbar}{2m}\pdv[2]{\zeta}{x}-\frac{i}{\hbar}\Bigl(V(x,t)+V_0\Bigr)\zeta\\
		&\implies \frac{i \hbar}{2m}\pdv[2]{\zeta}{x} - \frac{i V \zeta}{\hbar} = \pdv{\zeta}{t}+\frac{iV_0 \zeta}{\hbar}\\
	\end{flalign*}
	We consider the following boundary conditions: $\zeta\left(\infty, t\right) = \zeta \left( -\infty, t \right) =0$. We then proceed with letting $\zeta = X(x)T(t)$.
	\begin{flalign*}
		\zeta \left(x,t\right) &= X \left(x\right)T\left(t\right)\\
		\zeta\left(\infty,t\right)&=X\left(\infty \right)T\left(t\right)  ~~\forall t \implies c_1 = X(\infty)=0\\
		\zeta\left(-\infty,t\right)&=X\left(-\infty \right)T\left(t\right)  ~~\forall t \implies c_2 = X(-\infty)=0\\
		\frac{i \hbar}{2m}\pdv[2]{XT}{x}-\frac{iV XT}{\hbar} &= \pdv{XT}{t}+\frac{iV_0 XT}{\hbar}\\
		\frac{i \hbar}{2m} X''T-\frac{iV XT}{\hbar} &= XT'+\frac{iV_0 XT}{\hbar}\\
		-\frac{\hbar^2}{2m}\frac{X''}{X} +V &= \frac{T'}{T}i \hbar -V_0\\
		k\frac{X''}{X} &= \frac{T'}{T}i \hbar - V_0+V\\
		\frac{1}{k}(\frac{T'}{T}i \hbar-V_0+V) &= -\lambda\\
		-T(k \lambda -V_0 + V) &= T' i\hbar\\
		X'' &= -X\lambda \\
	\end{flalign*}
	For some arbitrary constant $\lambda$ and k. We now decompose into two ODEs, but first note he author saves us 
	here by hinting that $e^{-\frac{iV_0}{\hbar}}$ is a time-dependent phase factor we should expect to see in a solution.
	The implied boundary conditions would only give trivial solutions to the spatial eigenfunctions. For arbitrary constant C,
	\begin{flalign*}
		\zeta(x,t) = e^{\frac{i(k\lambda+V-V_0)}{\hbar}t+C} = \zeta_0 e^{\frac{i(V-V_0)}{\hbar}t} \\
	\end{flalign*}
	The author saves us here by the hint that $e^{-\frac{iV_0}{\hbar}t}$ is a time-dependent phase factor we should expect to see in a solution.
	The implied boundary conditions would only give trivial solutions to the spatial eigenfunctions. Thus, when plugging this back in to the wave equation
	we note the implication: $\Psi(x,t) = \zeta(x,t)e^{\frac{i V_0}{\hbar}t}$. If we substitute into equation 1.36, then we see that it remains unchanged. 
	We conclude that this has no effect on the expectation value of a dynamical variable, since the extra phase factor cancels out and is independent of position.
}